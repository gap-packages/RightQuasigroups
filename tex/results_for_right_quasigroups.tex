\documentclass{amsart}

\def\rdiv{/}
\def\ldiv{\backslash}
\def\rmlt#1{\mathrm{Mlt}_r(#1)}
\def\mlt#1{\mathrm{Mlt}(#1)}
\def\inn#1{\mathrm{Inn}(#1)}
\def\isordiv{/^*}
\def\isoldiv{\backslash^*}

\theoremstyle{plain}
\newtheorem{theorem}{Theorem}[section]
\newtheorem{acknowledgement}{Acknowledgement}
\newtheorem{axiom}{Axiom}%
\newtheorem{case}{Case}
\newtheorem{claim}{Claim}
\newtheorem{conclusion}{Conclusion}
\newtheorem{condition}[theorem]{Condition}
\newtheorem{conjecture}[theorem]{Conjecture}
\newtheorem{corollary}[theorem]{Corollary}
\newtheorem{criterion}{Criterion}
\newtheorem{definition}[theorem]{Definition}
\newtheorem{exercise}{Exercise}
\newtheorem{lemma}[theorem]{Lemma}
\newtheorem{notation}[theorem]{Notation}
\newtheorem{proposition}[theorem]{Proposition}
\newtheorem{observation}[theorem]{Observation}
\newtheorem{problem}[theorem]{Problem}
\newtheorem{solution}{Solution}
\newtheorem{summary}{Summary}
\newtheorem{construction}[theorem]{Construction}

\theoremstyle{definition}
\newtheorem{example}[theorem]{Example}
\newtheorem{remark}[theorem]{Remark}
\newtheorem{algorithm}[theorem]{Algorithm}

\begin{document}

\title{Some results used by the \texttt{GAP} package \texttt{RightQuasigroups}}

\author{G\'abor~P.~Nagy}

\author{Petr Vojt\v{e}chovsk\'y}

\maketitle

\section{Congruences}

Let $Q=(Q,\cdot,/)$ be a right quasigroup. Then an equivalence relation $\sim$ on $Q$ is a \emph{right quasigroup congruence} if for every $x,y,u,v\in Q$, if $x\sim y$ and $u\sim v$ then $xu\sim yv$ and $x/u\sim y/v$.

\begin{proposition}\label{Pr:RightQuasigroupCongruence}
Let $Q=(Q,\cdot,/)$ be a right quasigroup and $\sim$ an equivalence relation on $Q$. Then:
\begin{enumerate}
\item[(i)] $\sim$ is a right quasigroup congruence iff for every $x,y,u\in Q$, if $x\sim y$ then $xu\sim yu$, $x/u\sim y/u$, $ux\sim uy$ and $u/x\sim u/y$.
\item[(ii)] If $Q$ is finite then $\sim$ is a right quasigroup congruence iff for every $x,y,u\in Q$, if $x\sim y$ then $xu\sim yu$ and $ux\sim uy$.
\end{enumerate}
\end{proposition}
\begin{proof}
If $\sim$ is a right quasigroup congruence then certainly the conditions of (i) and (ii) hold. Conversely, suppose that the condition of (i) holds and let $x,y,u,v\in Q$ be such that $x\sim y$ and $u\sim v$. Then $xu\sim yu\sim yv$ and $x/u\sim y/u\sim y/v$ shows that $\sim$ is a right quasigroup congruence.

Finally suppose that $Q$ is finite and the condition of (ii) holds. We will verify the condition of (i). Suppose that $x,y,u\in Q$ and $x\sim y$. We then have $xu\sim yu$ and $ux\sim uy$ by assumption. Since $Q$ is finite, there is $n$ such that $R_u^n=1$ and thus $R_u^{-1}=R_u^{n-1}$. It follows by an easy induction on $n$ that $x/u = R_u^{-1}(x) = R_u^{n-1}(x) \sim R_u^{n-1}(y) = R_u^{-1}(y) = y/u$. Using finiteness again, let $s$ and $t$ be such that $R_x^s=1=R_y^t$. Consider $m = st-1$. Then $R_x^m = R_x^{st-1} = R_x^{-1}$ and $R_y^m = R_y^{ts-1} = R_y^{-1}$. We then again have $u/x = R_x^{-1}(u) = R_x^m(u)\sim R_y^m(u) = R_y^{-1}(u) = u/y$ by induction on $m$. The condition of (i) therefore holds and $\sim$ is a congruence.
\end{proof}

Let $Q=(Q,\cdot,/,\ldiv)$ be a right quasigroup. Then an equivalence relation $\sim$ on $Q$ is a \emph{quasigroup congruence} if for every $x,y,u,v\in Q$, if $x\sim y$ and $u\sim v$ then $xu\sim yv$, $x/u\sim y/v$ and $x\ldiv u \sim y\ldiv v$.

\begin{proposition}
Let $Q=(Q,\cdot,/,\ldiv)$ be a quasigroup and $\sim$ an equivalence relation on $Q$. Then:
\begin{enumerate}
\item[(i)] $\sim$ is a quasigroup congruence iff for every $x,y,u\in Q$, if $x\sim y$ then $xu\sim yu$, $ux\sim uy$, $x\rdiv u \sim y\rdiv u$ and $u\ldiv x\sim u\ldiv y$.
\item[(ii)] If $Q$ is finite then $\sim$ is a quasigroup congruence iff for every $x,y,u\in Q$, if $x\sim y$ then $xu\sim yu$ and $ux\sim uy$.
\end{enumerate}
\end{proposition}
\begin{proof}
If $\sim$ is a quasigroup congruence then the certainly the conditions of (i) and (ii) holds. Conversely, suppose that the condition of (i) holds and let $x,y,u,v\in Q$ be such that $x\sim y$ and $u\sim v$. Since $u\sim v$, we have $x = (x/u\cdot u)\sim (x/u\cdot v)$ and therefore $x/v \sim ((x/u\cdot v)/v) = x/u$. Also, from $x\sim y$ we get $x/v\sim y/v$. Therefore $x/u\sim x/v\sim y/v$. Dually, $x\ldiv u\sim y\ldiv v$. Hence $\sim$ is a quasigroup congruence.

If $Q$ is finite, the condition of (i) reduces to the condition of (ii) by the usual trick: $R_u^{-1}=R_u^{n-1}$ and $L_u^{-1}=L_u^{m-1}$ for suitable $n$ and $m$.
\end{proof}


\section{Simplicity}

Let $G$ be a group acting on $X$. Then $B\subseteq X$ is a \emph{block} of the action if for every $g\in G$ either $g(B)=B$ or $g(B)\cap B=\emptyset$. Given a partition $\mathcal P$ of $X$, we say that the action of $G$ \emph{preserves} $\mathcal P$ if for every $B\in\mathcal P$ and every $g\in G$ we have $g(B)\in\mathcal P$. The partitions $\{\{x\}:x\in X\}$ and $\{X\}$ are \emph{trivial}. A transitive permutation group $G$ acts \emph{primitively} on $X$ if it preserves no nontrivial partition of $X$, else it acts \emph{imprimitively}. (The requirement that $G$ be transitive is only needed if $|X|=2$.)

For a right quasigroup $Q$ let $\rmlt{Q}=\langle R_x:x\in Q\rangle$ be the \emph{right multiplication group} of $Q$. For a quasigroup $Q$ let $\mlt{Q}=\langle R_x,L_x:x\in Q\rangle$ be the \emph{multiplication group} of $Q$.

\begin{theorem}[Albert]
A quasigroup $Q$ is simple if and only if $\mlt{Q}$ acts primitively on $Q$.
\end{theorem}
\begin{proof}
Well known.
\end{proof}

\begin{example}
Consider the right quasigroup $Q$ with multiplication table
\begin{displaymath}
    \begin{array}{c|cccc}
         &1&2&3&4\\
        \hline
        1&2&1&1&1\\
        2&3&2&2&2\\
        3&4&3&3&3\\
        4&1&4&4&4
    \end{array}
\end{displaymath}
Then $G=\rmlt{Q} = \langle g\rangle$, where $g=(1,2,3,4)$. Note that $G$ acts transitively but imprimitively on $Q$, with $\{\{1,3\},\{2,4\}\}$ being a nontrivial partition of $Q$ preserved by $G$. However, an inspection of all possible partitions of $Q$ reveals that $Q$ has no nontrivial congruences and hence is simple. For instance, the above partition is not a right quasigroup congruence since $1\sim 3$ but $1\cdot 1 = 2\not\sim 1 = 1\cdot 3$.
\end{example}

\begin{proposition}
Let $Q$ be a right quasigroup. If $\rmlt{Q}$ acts primitively on $Q$ then $Q$ is simple. (The converse does not hold, as shown by the above example.)
\end{proposition}
\begin{proof}
Suppose that $Q$ is not simple and let $\sim$ be a nontrivial congruence on $Q$. Let $B$ be an equivalence class of $\sim$. If $y\sim z$ then $R_x(y)\sim R_x(z)$ and $R_x^{-1}(y)\sim R_x^{-1}(z)$ since $\sim$ is a congruence. In particular, $R_x(B)$ is contained in some equivalence class $C$ of $\sim$. Write $B=[b]$ and $C=[bx]$. If $c\in C$ then $c\sim bx$ and thus $c\rdiv x\sim (bx)\rdiv x = b$, so $c\rdiv x\in B$, but then $R_x(c\rdiv x) = (c\rdiv x)x = c$ shows that $R_x(B)=C$. Similarly, $R_x^{-1}(B)$ is an equivalence class of $\sim$. This shows that $\rmlt{Q}$ preserves the partition induced by $\sim$ and hence $\rmlt{Q}$ acts imprimitively on $Q$.
\end{proof}

\begin{lemma}
Let $Q$ be a right quasigroup. The orbits of $\rmlt{Q}$ form a right quasigroup congruence of $Q$.
\end{lemma}
\begin{proof}
Let $\sim$ be the equivalence relation induced by the orbits of $G=\rmlt{Q}$. Suppose that $x\sim y$ and $u\in Q$. Then $ux = R_x(u)\sim R_y(u) = uy$ and $u/x = R_x^{-1}(u)\sim R_y^{-1}(u) = u/y$. Let $g\in G$ be such that $g(x)=y$. Then $xu = R_u(x)\sim R_u(g(x)) = R_u(y) = yu$ and $x/u = R_u^{-1}(x)\sim R_u^{-1}(g(x)) = R_u^{-1}(y) = y/u$. By Proposition \ref{Pr:RightQuasigroupCongruence}, $\sim$ is a right quasigroup congruence.
\end{proof}

\begin{corollary}
Let $Q$ be a right quasigroup and suppose that $\rmlt{Q}\ne 1$ does not act transitively on $Q$. Then $Q$ is not simple.
\end{corollary}

Note that a right quasigroup $Q$ satisfies $\rmlt{Q}=1$ if and only if it is a projection right quasigroup, that is, a right quasigroup with multiplication and right division given by $xy=x$, $x/y=x$.

\begin{lemma}
Let $Q$ be a projection right quasigroup. Then any partition of $Q$ is a right quasigroup congruence of $Q$. In particular, $Q$ is simple if and only if $|Q|>2$.
\end{lemma}
\begin{proof}
Let $\sim$ be the equivalence relation induced by a given partition of $Q$. Suppose that $x\sim y$ and $u\in Q$. Then $xu = x\sim y = yu$, $x/u = x\sim y = y/u$, $ux = u\sim u = uy$ and $u/x = u\sim u = u/y$. By Proposition \ref{Pr:RightQuasigroupCongruence}, $\sim$ is a right quasigroup congruence.
\end{proof}

\section{Nuclei and center}

\begin{proposition}\label{Pr:Sub}
A nonempty subset $S$ of a finite (right) quasigroup $Q$ is a sub(right)quasigroup of $Q$ iff it is closed under multiplication.
\end{proposition}
\begin{proof}
In the case of right quasigroups, it suffices to show that $S$ is closed under right division. For $x,y\in S$, consider $R_x\in\mathrm{Sym}(Q)$. Since $Q$ is finite, there is $n$ such that $R_x^n=\mathrm{id}_Q$, so $R_x^{-1} = R_x^{n-1}$. Then $y\rdiv x = R_x^{-1}(y) = R_x^{n-1}(y)\in S$ by induction on $n$. The argument for left divisions is dual in the case of quasigroups.
\end{proof}

\begin{proposition}
Let $Q$ be a finite (right) quasigroup. Then each of the four nuclei is either a sub(right)quasigroup of $Q$ or the empty set.
\end{proposition}
\begin{proof}
Let $S=\mathrm{Nuc}_\ell(Q)\ne\emptyset$. Then for every $x,y\in S$ and every $u,v\in Q$ we have $(xy)(uv) = x(y(uv)) = x((yu)v) = (x(yu))v = ((xy)u)v$, so $xy\in S$ and we are done by Proposition \ref{Pr:Sub}. Dually, if $\mathrm{Nuc}_r(Q)\ne\emptyset$ then it is a sub(right)quasigroup of $Q$. Now suppose that $S=\mathrm{Nuc}_m(Q)\ne\emptyset$. Then for all $x,y\in S$ and $u,v\in Q$ we have $(u(xy))v = ((ux)y)v = (ux)(yv) = u(x(yv)) = u((xy)v)$, so $xy\in S$ and we are done by Proposition \ref{Pr:Sub}. The intersection of sub(right)quasigroups is a sub(right)quasigroup.
\end{proof}

\begin{proposition}
Let $Q$ be a finite (right) quasigroup. Then the center of $Q$ is either a sub(right)quasigroup of $Q$ or the empty set. (Do we need finiteness here?)
\end{proposition}
\begin{proof}
It remains to prove that if $x,y\in Z(Q)$ and $u\in Q$ then $(xy)u=u(xy)$. We have $(xy)u = x(yu) = (yu)x = (uy)x = u(yx) = u(xy)$.
\end{proof}

\section{Lower central series for loops}

The lower central series for a loop $Q$ is defined by $Q_{(0)}=Q$, $Q_{(i+1)}=[Q_{(i)},Q]_Q$, using the congruence commutator of normal subloops. Here we are only using the commutator of the form $[A,Q]_Q$ for $A\unlhd Q$. It's easy to see that $[A,Q]_Q=D$ iff $D$ is the smallest normal subloop of $Q$ such that $A/D\le Z(Q/D)$.

\begin{lemma}
Let $A\unlhd Q$. Then $[A,Q]_Q$ is the smallest normal subloop of $Q$ containing $\{\theta(a)/a:a\in A,\,\theta\in\inn{Q}\}$.
\end{lemma}
\begin{proof}
Let $D\unlhd Q$. The following conditions are equivalent:
\begin{itemize}
\item $A/D\le Z(Q/D)$
\item $\theta(aD)=aD$ for all $a\in A$, $\theta\in\inn{Q/D}$
\item $L_{xD,yD}(aD)=aD$, $R_{xD,yD}(aD)=aD$, $T_{xD}(aD)=aD$ for all $x,y\in Q$, $a\in A$
\item $L_{x,y}(a)D = aD$, $R_{x,y}(a)(D)=aD$, $T_x(a)D=aD$ for all $x,y\in Q$, $a\in A$,
\item $\theta(a)D=aD$ for all $a\in A$, $\theta\in\inn{Q}$
\item $\theta(a)/a\in D$ for all $a\in A$, $\theta\in\inn{Q}$.
\end{itemize}
\end{proof}

\section{Displacement groups}

For a right quasigroup $(Q,\cdot)$, define the \emph{right positive displacement group}, the \emph{right negative displacement group} and the \emph{right displacement group} by
\begin{align*}
    \mathrm{Dis}_r^+(Q) &=\langle R_xR_y^{-1}:x,y\in Q\rangle,\\
    \mathrm{Dis}_r^-(Q) &=\langle R_x^{-1}R_y:x,y\in Q\rangle,\\
    \mathrm{Dis}_r(Q) &= \langle R_xR_y^{-1},R_x^{-1}R_y:x,y\in Q\rangle,
\end{align*}
respectively.

Fix $e\in Q$. Since $R_xR_y^{-1} = (R_eR_x^{-1})^{-1}(R_eR_y^{-1}) = (R_xR_e^{-1})(R_yR_e^{-1})^{-1}$ and $R_x^{-1}R_y=(R_x^{-1}R_e)(R_y^{-1}R_e)^{-1} = (R_e^{-1}R_x)^{-1}(R_e^{-1}R_y)$, we have
\begin{align*}
    \mathrm{Dis}_r^+(Q)=\langle R_eR_x^{-1}:x\in Q\rangle = \langle R_xR_e^{-1}:x\in Q\rangle,\\
    \mathrm{Dis}_r^-(Q)=\langle R_x^{-1}R_e:x\in Q\rangle = \langle R_e^{-1}R_x:x\in Q\rangle.
\end{align*}

The left displacement groups are defined analogously for a left quasigroup $(Q,\cdot)$ by
\begin{align*}
    \mathrm{Dis}_\ell^+(Q) &=\langle L_xL_y^{-1}:x,y\in Q\rangle,\\
    \mathrm{Dis}_\ell^-(Q) &=\langle L_x^{-1}L_y:x,y\in Q\rangle,\\
    \mathrm{Dis}_\ell(Q) &= \langle L_xL_y^{-1},L_x^{-1}L_y:x,y\in Q\rangle,
\end{align*}
and we once again have
\begin{align*}
    \mathrm{Dis}_\ell^+(Q)=\langle L_eL_x^{-1}:x\in Q\rangle = \langle L_xL_e^{-1}:x\in Q\rangle,\\
    \mathrm{Dis}_\ell^-(Q)=\langle L_x^{-1}L_e:x\in Q\rangle = \langle L_e^{-1}L_x:x\in Q\rangle
\end{align*}
for a fixed $e\in Q$.

\begin{proposition}
Let $(Q,\cdot)$ be a quasigroup. Then $(Q,\cdot)$ is isotopic to a group if and only if the left positive displacement group $\mathrm{Dis}_\ell^+(Q,\cdot)$ acts regularly on $Q$. In that case, $(Q,\cdot)$ is isotopic to
$\mathrm{Dis}_\ell^+(Q,\cdot)$.
\end{proposition}
\begin{proof}
Let $D=\mathrm{Dis}_\ell^+(Q,\cdot)$. Given $y,z\in Q$, there exists a unique $x\in Q$ such that $L_xL_e^{-1}(y)=z$, namely $x = z\rdiv (e\ldiv y)$. Suppose that $D$ acts regularly on $Q$. Then $D=\{L_xL_e^{-1}:x\in Q\}$ and for every $x,y\in Q$ there is $z\in Q$ such that $L_xL_e^{-1}L_yL_e^{-1} = L_zL_e^{-1}$. Thus $L_xL_e^{-1}L_y = L_z$ and, applying this to $e$, we get $x(e\ldiv (ye)) = ze$ and $z=x(e\ldiv ye)\rdiv e$. Define $(Q,*)$ by $x*y = x(e\ldiv ye)\rdiv e$. Then $f:D\to (Q,*)$, $L_xL_e^{-1}\mapsto x$ is an isomorphism, so $(Q,*)$ is a group. Since $(x*y)e = x(e\ldiv ye)$, the triple $(\mathrm{id},L_e^{-1}R_e,R_e)$ is an isotopism $(Q,*)\to (Q,\cdot)$.

Conversely, suppose that $(Q,*)$ is a group and $(\alpha,\beta,\gamma)$ is an isotopism $(Q,*)\to (Q,\cdot)$, so $\alpha(x)\cdot\beta(y) = \gamma(x*y)$, or $x\cdot y = \gamma(\alpha^{-1}(x)*\beta^{-1}(y))$ for all $x,y\in Q$. This shows that the left translation by $x$ in $(Q,\cdot)$ is equal to $L_x = \gamma L_{\alpha^{-1}(x)}^*\beta^{-1}$. Then
\begin{align*}
    L_xL_e^{-1} &= (\gamma L_{\alpha^{-1}(x)}^*\beta^{-1})(\gamma L_{\alpha^{-1}(e)}^*\beta^{-1})^{-1}\\
     &= \gamma L_{\alpha^{-1}(x)}^*(L_{\alpha^{-1}(e)}^*)^{-1}\gamma^{-1} = \gamma L_{\alpha^{-1}(x)*(\alpha^{-1}(e))^{-1}}\gamma^{-1}
\end{align*}
because $(Q,*)$ is a group. Hence $D$ is a conjugate of $\langle L_{\alpha^{-1}(x)*(\alpha^{-1}(e))^{-1}}:x\in Q\rangle = \langle L_x^*:x\in Q\rangle = \{L_x^*:x\in Q\}$, which certainly acts regularly on $Q$.
\end{proof}

\begin{corollary}
A quasigroup $Q$ is isotopic to a group iff $|\mathrm{Dis}_\ell^+(Q)|=|Q|$.
\end{corollary}

\section{Twists of right quasigroups}

Given a magma $Q$ and three mappings $f,g,h:Q\to Q$, the \emph{twist} $\mathrm{Tw}(Q,f,g,h)$ of $Q$ via $(f,g,h)$ is defined to be the magma $(Q,*)$ with multiplication $x*y = h(f(x)g(y))$.

If $Q$ is a right quasigroup, the twist $\mathrm{Tw}(Q,f,g,h)$ is a right quasigroup iff both $f$ and $h$ are bijections of $Q$. Moreover, the twist $\mathrm{Tw}(Q,f,g,h)$ is a quasigroup iff all three $f$, $g$ and $h$ are bijections of $Q$. Finally, if $\mathrm{Tw}(Q,f,g,h)$ is a quasigroup then it is a loop iff $g^{-1}(f(x)\backslash h^{-1}(x))$ is equal to $f^{-1}(h^{-1}(x)/g(x))$ and independent of $x$.

Isotopes and affine constructions can be realized as twists.

\section{Affine right quasigroups}

Given a loop $(Q,\cdot)$, its automorphism $f$, endomorphism $g$ and two elements $u$, $v$, define $\mathrm{Aff}(Q,\cdot,f,u,g,v)=(Q,*)$ by $x*y = (f(x)u)(g(y)v)$. (We also allow variations with $uf(x)$, $vg(y)$ and any combinations. For instance, $\mathrm{Aff}(Q,\cdot,u,f,g,v)$ has multiplication $x*y = (uf(x))(g(y)v)$.) Then $(Q,*)$ is \emph{affine} over $(Q,\cdot)$ and $(Q,\cdot,f,u,g,v)$ is the \emph{arithmetic form} of $(Q,*)$.

\begin{lemma}
$(Q,*)$ is a right quasigroup. $(Q,*)$ is a quasigroup iff $g$ is an automorphism.
\end{lemma}
\begin{proof}
Solving $x*y = (f(x)u)(g(y)v)=z$ for $x$ yields $x = f^{-1}((z/(g(y)v))/u)$ and similarly in the other three cases. Solving $x*y=(f(x)u)(g(y)v)=z$ for $y$ is equivalent to solving $g(y) = ((f(x)u)\ldiv z)\rdiv v$.
\end{proof}

If $(Q,\cdot)$ is an abelian group, the formula $x*y = (f(x)u)(g(y)v)$ becomes $f(x)g(y)uv$ and it therefore suffices to consider only arithmetic forms $(Q,\cdot,f,g,c)$ with automorphism $f$, endomorphism $g$ and central element $c$, and define the multiplication by $x*y =f(x)g(y)c$. In general this is a special case of the affine construction. From now on we assume that we are dealing with the special case $(Q,f,g,c)$.

\begin{lemma}\label{Lm:AffineRack}
$(Q,*)$ is a rack iff $g(c)=1$, $fg=gf$, $g(x)=fg(x)g^2(x)$ and $xfg(y)\cdot g(z) = xfg(z)\cdot fg(y)g^2(z)$ for all $x,y,z\in Q$.
\end{lemma}
\begin{proof}
We have $(x*y)*z = (f(x)g(y)c)*z = f^2(x)fg(y)f(c)\cdot g(z)\cdot c$, while $(x*z)*(y*z) = (f(x)g(z)c)*(f(y)g(z)c) = f^2(x)fg(z)f(c)\cdot gf(y)g^2(z)g(c)\cdot c$. Since $c$, $f(c)$ and $g(c)$ are central, we see that $(Q,*)$ is a rack iff $f^2(x)fg(y)\cdot g(z) = f^2(x)fg(z)\cdot gf(y)g^2(z)\cdot g(c)$. Substituting $x=y=z=1$ then yields $g(c)=1$ as a necessary condition. Assuming this, we need to verify $f^2(x)fg(y)\cdot g(z) = f^2(x)fg(z)\cdot gf(y)g^2(z)$. With $x=z=1$ we obtain $fg(y) = gf(y)$ as a necessary condition. Assuming this, we need to verify $f^2(x)fg(y)\cdot g(z) = f^2(x)fg(z)\cdot fg(y)g^2(z)$. With $x=y=1$ we get $g(z)=fg(z)g^2(z)$ as a necessary condition. Assuming this and substituting $x$ for $f^2(x)$ yields the last condition.
\end{proof}

Substituting $fg(z)g^2(z)$ for $g(z)$ into the left hand side of the last condition of Lemma \ref{Lm:AffineRack} yields $xfg(y)\cdot fg(z)g^2(z) = xfg(z)\cdot fg(y)g^2(z)$. This condition is certainly satisfied when $(Q,\cdot)$ is a medial loop. Recall that a loop is medial iff it is an abelian group. Indeed, from $(xu)(vy)= (xv)(uy)$ we obtain commutativity with $x=y=1$ and associativity with $v=1$.

\begin{lemma}\label{Lm:AffineQuandle}
$(Q,*)$ is a quandle iff $c=1$, $g(x)=f(x)\ldiv x$ and $xfg(y)\cdot g(z) = xfg(z)\cdot fg(y)g^2(z)$ for all $x,y,z\in Q$.
\end{lemma}
\begin{proof}
We have $x*x=x$ iff $f(x)g(x)c=x$. Substituting $x=1$ yields $c=1$. Using this, we have $x*x=x$ iff $f(x)g(x)=x$, that is, $g(x)=f(x)\ldiv x$. If $g(x)=f(x)\ldiv x$ then both $fg=gf$ and $g(x)=fg(x)g^2(x)$ hold. We are done by Lemma \ref{Lm:AffineRack}.
\end{proof}

Note that the conditions of Lemma \ref{Lm:AffineQuandle} do not impose any restrictions on the loop $(Q,\cdot)$. Indeed, if $(Q,\cdot)$ is any loop, $f(x)=x$, $g(x)=1$ and $c=1$ then $x*y=f(x) = x$ and $(Q,*)$ is a projection quandle.

Also note that a latin rack is a quandle. Indeed, substituting $z=y$ into $(xy)z=(xz)(yz)$ yields $(xy)y = (xy)(yy)$ and canceling $xy$ on the left then yields $y=yy$.

\begin{lemma}\label{Lm:AffineLatinQuandle}
$(Q,*)$ is a latin rack (i.e., latin quandle) iff $c=1$, $g(x)=f(x)\ldiv x$ and $xy\cdot z = xf(z)\cdot y(f(z)\ldiv z)$ for all $x,y,z\in Q$.
\end{lemma}
\begin{proof}
Replace $g(z)$ with $z$ and $fg(y)$ with $y$ in the last condition of Lemma \ref{Lm:AffineQuandle}.
\end{proof}

\begin{corollary}
Suppose that $(Q,\cdot)$ is an abelian group. Then $(Q,*)$ is a rack iff $g(c)=1$, $fg=gf$ and $g(x)=fg(x)g^2(x)$ for all $x\in Q$.
\end{corollary}
\begin{proof}
Suppose that $g(x)=fg(x)g^2(x)$ for all $x\in Q$. Then $xfg(y)\cdot g(z) = xfg(y)\cdot fg(z)g^2(z) =  xfg(z)\cdot fg(y)g^2(z)$, where we have used mediality in the last step.
\end{proof}

\begin{corollary}
Suppose that $(Q,\cdot)$ is an abelian group. Then $(Q,*)$ is a quandle iff $c=1$ and $g(x)=f(x)\ldiv x = xf(x)^{-1}$.
\end{corollary}

\section{Calculating isotopisms and autotopisms}

Let $Q_1 = (Q_1,\cdot,\ldiv,\rdiv)$, $Q_2=(Q_2,*,\isoldiv,\isordiv)$ be quasigroups. The triple $(f,g,h)$ of mappings $Q_1\to Q_2$ is a homotopism if $f(x)*g(y) = h(x\cdot y)$ for all $x,y\in Q_1$.

\begin{lemma}
Let $c$ be a fixed element of $Q_1$. A homotopism $(f,g,h)$ from $Q_1$ to $Q_2$ is determined by the values of one of the three mappings on $Q_1$ and by the value on $c$ of one of the two remaining mappings.
\end{lemma}
\begin{proof}
We will give a proof when $h(x)$ is known for all $x\in Q_1$ and $f(c)$ is known. The remaining five cases are similar.
We have $f(c)*g(c\ldiv x) = h(c(c\ldiv x)) = h(x)$ and hence $g(c\ldiv x) = g(c)\isoldiv h(x)$. This shows that $g(x)$ is determined for all $x\in Q_1$. We also have $f(x\rdiv c)*g(c) = h((x\rdiv c)c) = h(x)$ and hence $f(x\rdiv c) = h(x)\isordiv g(c)$. This shows that $f(x)$ is determined for all $x\in Q_1$.
\end{proof}

The following result shows how the domain of a partially defined homotopism must be extended (iteratively) whenever a new image of $f$, $g$ or $h$ has been chosen. The domain of a mapping $f$ is denoted by $D(f)$.

\begin{lemma}
Let $(f,g,h)$ be a partial homotopism $Q_1\to Q_2$.
\begin{enumerate}
\item[(i)] If $x\in D(f)$ then $g(x\ldiv y) = f(x)\isoldiv h(y)$ for all $y\in D(h)$ and $h(xy) = f(x)*g(y)$ for all $y\in D(g)$.
\item[(ii)] If $x\in D(g)$ then $f(y\rdiv x) = h(y)\isordiv g(x)$ for all $y\in D(h)$ and $h(yx) = f(y)*g(x)$ for all $y\in D(f)$.
\item[(iii)] If $x\in D(h)$ then $g(y\ldiv x) =f(y)\isoldiv h(x)$ for all $y\in D(f)$ and $f(x\rdiv y) = h(x)\isordiv g(y)$ for all $y\in D(g)$.
\end{enumerate}
\end{lemma}

\end{document} 